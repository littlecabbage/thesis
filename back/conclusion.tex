% !TEX root = ../main.tex

% 结论
\begin{conclusions}

    % 学位论文的结论作为论文正文的最后一章单独排写,但不加章标题序号。
    
    % 结论应是作者在学位论文研究过程中所取得的创新性成果的概要总结,不能与摘要混为一谈。博士学位论文结论应包括论文的主要结果、创新点、展望三部分,在结论中应概括论文的核心观点,明确、客观地指出本研究内容的创新性成果(含新见解、新观点、方法创新、技术创新、理论创新),并指出今后进一步在本研究方向进行研究工作的展望与设想。对所取得的创新性成果应注意从定性和定量两方面给出科学、准确的评价,分(1)、(2)、(3)…条列出,宜用“提出了”、“建立了”等词叙述。
    时间序列异常检测是工业界中一个重要的研究领域。尤其随着近年来工业4.0概念的提出,越来越多的行业提倡数字化,智能化,这激增了海量的时间序列数据。时间序列异常检测吸引了越来越多的研究者的关注,近年来有诸多学者提出了非常多的多元时间序列异常检测的算法及模型。然而这些方法都是都是面向完整的时间序列数据进行异常检测任务,然而在真实时间中多元时间序列往往包含着大量的缺失值。例如在工业场景中常常因为数据采集不全,传感器损坏等诸多因素所导致时间序列中包含缺失值。故本文课题为缺失值场景下的多元时间序列异常检测,按照填充-检测和不填充-检测的两种思路对包含缺失值的多元时间序列异常检测任务进行了如下研究:
    
    (1)按照填充-检测的思路,本文首先提出了一种基于对抗生成网络的填充算法,并对填充后的多元时间序列进行异常检测。较好的解决缺失值场景下的多元时间序列异常检测任务。本方法通过与多个基线模型对比,在完整数据集上,本章F1分数超过了第二名的基线0.3\%在缺失30\%数据的场景下F1分数超过了第二名的基线3\%,在缺失50\%数据的场景下F1分数超过了第二名的基线5\%。证明了我们的算法在缺失值场景的有效性。
    
    (2)考虑到在填充-检测的解决方案中,模型在训练时需要同时包含缺失值的时间序列和不包含缺失值的时间序列进行学习。这对数据的要求较高,在现实应用中实现难度较大。填充-检测的思路方案依然不能完美的解决缺失值多元时间序列异常检测问题。故本文提出了一种不填充-检测思路的多元时间序列异常检测方案,其基于注意力机制对缺失值时间序列进行重新表征,将不完整的多元时间序列重新表征为完整的高维表征,进而通过对该表征进行异常检测,更鲁棒的解决缺失值场景下的异常检测任务。本方法在三个常用经典时间序列数据集上进行的实验表明,该方法在缺失值场景下对比传统时间序列异常检测方法效果更好。最后,本文通过消融实验验证了各模块的有效性。
    
    本文的研究工作仍然存在着一定不足。例如当前异常检测的数据源过于单一,本文所提出的方法只能识别时间序列中的异常。 未来可以考虑引入多任务学习框架来联合现实世界中日志,图片等信息共同完成异常检测。同时,本文的异常检测模型在可解释性方面不足,未来可考虑引入根因分析模块以提高模型的可解释性帮助使用者完成根因定位。
    
    \end{conclusions}
    